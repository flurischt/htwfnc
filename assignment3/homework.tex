\documentclass[10pt,a4paper,oneside,notitlepage]{report}
\usepackage[utf8]{inputenc}
\usepackage[english]{babel}
\usepackage{amsmath}
\usepackage{amsfonts}
\usepackage{amssymb}
\usepackage{hyperref}
\usepackage[table]{xcolor}  
\usepackage[margin=2.5cm]{geometry}
\usepackage{titling}
\usepackage{titlesec}
\usepackage{graphicx}
\usepackage{float}

% kein einrücken bei neuen absätzen
\setlength{\parindent}{0pt}

% position titel
\setlength{\droptitle}{-2cm}

% abstand vor subsection titel
\titlespacing{\subsection}{0pt}{40pt}{8pt}

\author{Flurin Rindisbacher}
\title{How To Write Fast Numerical Code \\ \vspace{6 mm} \textbf{Assignment X}}

\begin{document}
\maketitle

\section*{Exercise 1 - Cache mechanics}
\subsection*{(a)}
There are $32KB / 64 Byte = 512$ cache blocks. The number of sets is therefore $\underline{512 / 8=64}$.

\subsection*{(b)}
The block size is $64$ bytes and with $64=2^6$ 6 bits are needed to determine the position in the block. \\
Also 6 bits are needed for the set. \\
\underline{This results in a tag length of $64 - 6 - 6=52$ bits.}
\textbf{TODO:} effectly used on this machine??

\subsection*{(c)}
$0xDE147BA$ results ins the following values: \\
tag: $0xDE14$ \\
set: $0x1E$ \\
block: $0x3A$
\end{document}
