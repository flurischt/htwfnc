\documentclass[10pt,a4paper,oneside,notitlepage]{report}
\usepackage[utf8]{inputenc}
\usepackage[english]{babel}
\usepackage{amsmath}
\usepackage{amsfonts}
\usepackage{amssymb}
\usepackage{hyperref}
\usepackage[table]{xcolor}  
\usepackage[margin=2.5cm]{geometry}
\usepackage{titling}
\usepackage{titlesec}
\usepackage{graphicx}
\usepackage{float}

% kein einrücken bei neuen absätzen
\setlength{\parindent}{0pt}

% position titel
\setlength{\droptitle}{-2cm}

% abstand vor subsection titel
\titlespacing{\subsection}{0pt}{40pt}{8pt}

\author{Flurin Rindisbacher}
\title{How To Write Fast Numerical Code \\ \vspace{6 mm} \textbf{Assignment 3}}

\begin{document}
\maketitle

\section*{Exercise 1 - Cache mechanics}
\subsection*{(a)}
There are $32KB / 64 Byte = 512$ cache blocks. The number of sets is therefore $\underline{512 / 8=64}$.

\subsection*{(b)}
The block size is $64$ bytes and with $64=2^6$ 6 bits are needed to determine the position in the block. \\
Also 6 bits are needed for the set. \\
\underline{This results in a tag length of $64 - 6 - 6=52$ bits.}
\\
\textbf{TODO:} effectly used on this machine??

\subsection*{(c)}
$0xDE147BA$ results ins the following values: \\
tag: $0xDE14$ \\
set: $0x1E$ \\
block: $0x3A$

\section*{Exercise 2 - Cache mechanics}
cache size: 32KB \\
block size: 64 bytes \\
number of sets: 32KB / 16bytes = 1024 sets \\
because of sizeof(int) = 4, 4 integers fit into one block. \\
address of src matrix: 0x00 \\
address of dest matrix assuming size 64x64: 0x4000 ($64*64*4=16384$) \\
address of dest matrix assuming size 96x64 = 0x6000 ($96*64*4=25576$) \\
Number of bits for block: 4 \\
Number of bits for set: 10\\
\subsection*{(a)}
\subsubsection*{i}
with 4 integers per block and 64 ints per row there are 16 blocks to read/write per row. 0x00 (src) and 0x4000 (dest) map to the same cache set. we therefore know, that they'll use the same 16 blocks for the dest[i] and src[i] row.  \\ Assuming the used blocks are numbered from 0 to 15 the access pattern will be the following: \\

\begin{tabular}{|c|c|c|c|}
\hline 
\rowcolor{gray!30}
\textbf{block for src} &\textbf{ block for dest} & \textbf{hit/miss src} & \textbf{hit/miss dest} \\ 
\hline 
0 & 15 & MHHH & MHHH \\ 
\hline 
1 & 14 & MHHH & MHHH \\ 
\hline 
2 & 13 & MHHH & MHHH \\ 
\hline 
3 & 12 & MHHH & MHHH \\ 
\hline 
4 & 11 & MHHH & MHHH \\ 
\hline 
5 & 10 & MHHH & MHHH \\ 
\hline 
6 & 9 & MHHH & MHHH \\ 
\hline 
7 & 8 & MHHH & MHHH \\ 
\hline 
8 & 7 & MHHH & MHHH \\ 
\hline 
9 & 6 & MHHH & MHHH \\ 
\hline 
10 & 5 & MHHH & MHHH \\ 
\hline 
11 & 4 & MHHH & MHHH \\ 
\hline 
12 & 3 & MHHH & MHHH \\ 
\hline 
13 & 2 & MHHH & MHHH \\ 
\hline 
14 & 1 & MHHH & MHHH \\ 
\hline 
15 & 0 & MHHH & MHHH \\ 
\hline 
\end{tabular} 

Because the src / dest never uses the same block at the same time we always get an access pattern of MHHH. This results in a cache miss rate of $\underline{1/4=0.25}$

\subsubsection*{ii}
0x00 (src) points to set number 0. 0x6000 (dest) points to set number 512.
So when src reads the cache block $b=(i \mod 1024)$ dest uses block $b=(i+512 \mod 1024)$. This always gives a MHHH access pattern for read and write. In total thats a cache miss rate of $\underline{1/4=0.25}$

\subsection*{(b)}
\subsubsection*{i}
for each column 64 blocks for dest and 64 blocks for src have to be loaded. starting at i=0, j=0 src would use the cache blocks 0:16:1008 (matlab notation, 16 as the stepsize) and dest would use the cache blocks 15:16:1023. after loading the block they can be used for the following three columns leading to a MHHH pattern for four columns. this also gives us a cache miss rate of $\underline{1/4=0.25}$

\subsubsection*{ii}
using the same notation as in i) while processing the first column the first 64 rows would be loaded into the blocks 0:16:1008 (for src). Since we have now 96 rows the next 32 will be loaded into the blocks 0:16:496. This overwrites the first 32 blocks that we have loaded before.   This means, that the first and last 32 rows would always overwrite each others block. The blocks loaded for the rows 32 to 63 can be reused for the 3 columns processed afterwards (as before in i). \\
The following table shows the hit/miss pattern when 4 columns are processed:\\
\begin{tabular}{|c|c|c|}
\hline 
\rowcolor{gray!30}
\textbf{rows}  & \textbf{used cache blocks} & \textbf{hit/miss} \\ 
\hline 
0 to 31 & 0 to 1008 with a step of 16 & MMMM \\ 
\hline 
32 to 63 & 15 to 1023 with a step of 16 & MHHH \\ 
\hline 
64 to 95 & 0 to 1008 with a step of 16 & MMMM \\ 
\hline 
\end{tabular}  \\
This results in a cache miss rate of $\underline{9/12=0.75}$

\section*{Exercise 3 - Cache mechanics}
\subsection*{(a)}
\subsubsection*{i}
Four photon\_t fit into one cache block. For four consecutive photon\_t the write to the first irr[0] will miss. Writes to the remaining irr[1], irr[2], theta, phi and all writes for the remaining three photon\_t will hit. 1 miss, 4 hits for the first photon\_t and 3*5 hits for the remaining three gives a cache miss rate of $\underline{1/20=0.05}$
\subsubsection*{ii}
Two photon\_t fit into one block. This results in a MHMHMH pattern (when looking just at the number of photon\_t). Couting the writes this gives a cache miss rate of $\underline{1/10=0.1}$
\end{document}
