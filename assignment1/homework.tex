\documentclass[10pt,a4paper,oneside]{report}
\usepackage[utf8]{inputenc}
\usepackage[english]{babel}
\usepackage{amsmath}
\usepackage{amsfonts}
\usepackage{amssymb}
\usepackage{hyperref}
\usepackage[table]{xcolor}  
\usepackage[top=2cm,bottom=4cm,left=1.5cm,right=3cm,asymmetric]{geometry}
\author{Flurin Rindisbacher}
\title{How To Write Fast Numerical Code Assignment 1}
\begin{document}
\chapter*{Assignment 1}
How To Write Fast Numerical Code \\
Flurin Rindisbacher, rflurin@student.ethz.ch \\
Spring Term 2015
\section*{Exercise 1 - Get to know your machine}
I own a MacBook Pro (Retina, Mid 2012) 2.6 GHz Intel Core i7. According to \textit{sysctl -n machdep.cpu.brand\_string} and the Intel manuals the microarchitectural parameters are: \\

\begin{tabular}{|l|l|l|}
\hline 
\rowcolor{gray!30}
\textbf{Parameter} & \textbf{Value} & \textbf{Assignment} \\ 
\hline 
Processor manufacturer & Intel & a)\\ 
\hline 
Processor name & Core i7 & a)\\ 
\hline
Processor number & 3720QM  & a)\\ 
\hline
\# of cores & 4  & b) \\ 
\hline 
CPU-core frequency & 2.60GHz  & c)\\ 
\hline 
Tick/tok & It's a Tick model (Ivy Bridge, shrink to 22nm) & d) \\ 
\hline 
Cycles/issue for floating point additions & 1 add/cycle & e) \\ 
\hline 
Cycles/issue for floating point multiplications & 1 mul/cycle & f) \\ 
\hline 
Max. theoretical float peak performance & 2 flops/cycle and 5.2 Gflop/s & g) \\ 
\hline 
\end{tabular} 
\\
Source of Information:
\begin{itemize}
  \item \url{http://www.intel.com/content/dam/doc/manual/64-ia-32-architectures-optimization-manual.pdf} Page 2-37
  \item \url{http://ark.intel.com/products/64891/Intel-Core-i7-3720QM-Processor-6M-Cache-up-to-3_60-GHz}
  \item ``sysctl -n machdep.cpu.brand\_string''
\end{itemize}

\section*{Exercise 2 - Cost analysis}
\subsection*{a)}
let $C(N) = float\_muls(N) + float\_adds (N) + float\_divs(N) + float\_mins(N)$ with \\
$float\_muls = $ number of floating point multiplications \\
$float\_adds = $number of floating point additions \\
$float\_divs = $number of floating point divisions calls \\
$float\_mins = $number of floating point min() calls \\

\subsection*{b)}
There are two inner-most loops containing operations in the function \textit{strong\_closure}. Both of these for-loops are nested inside two other loops going from 0..N/2-1 and 0..N-1. Since both loops go from 0 to N-1 their bodies are executed: $N/2 * N * N = N^3/2$ times.
\\
$float\_divs = N^3/2$ \\
$float\_muls = 0$ \\
$float\_adds = 7*N^3/2$ \\
$float\_mins = 5*N^3/2$\\
and therefore $C(N) = N^3/2 float\_divs + 7*N^3/2 float\_adds + 5*N^3/2 float\_mins$ for the function strong\_closure()

\subsection*{c)}
I would define $C(N) = {\#flops}$. This would lead to a cost of $C(N) = 13*N^3/2 = 7.5*N^3$ for the strong\_closure() function.

\section*{Exercise 3 - Floyd Warshall}
asdf

\end{document}