\documentclass[10pt,a4paper,oneside]{report}
\usepackage[utf8]{inputenc}
\usepackage[english]{babel}
\usepackage{amsmath}
\usepackage{amsfonts}
\usepackage{amssymb}
\usepackage{hyperref}
\usepackage[table]{xcolor}  
\usepackage[top=2cm,bottom=4cm,left=1.5cm,right=3cm,asymmetric]{geometry}
\author{Flurin Rindisbacher}
\title{How To Write Fast Numerical Code Assignment 1}
\begin{document}
\chapter*{Assignment 1}
How To Write Fast Numerical Code \\
Flurin Rindisbacher, rflurin@student.ethz.ch \\
Spring Term 2015
\section*{Exercise 1}
I own a MacBook Pro (Retina, Mid 2012) 2.6 GHz Intel Core i7. According to \textit{sysctl -n machdep.cpu.brand\_string} and the Intel manuals the microarchitectural parameters are: \\

\begin{tabular}{|l|l|l|}
\hline 
\rowcolor{gray!30}
\textbf{Parameter} & \textbf{Value} & \textbf{Assignment} \\ 
\hline 
Processor manufacturer & Intel & a)\\ 
\hline 
Processor name & Core i7 & a)\\ 
\hline
Processor number & 3720QM  & a)\\ 
\hline
\# of cores & 4  & b) \\ 
\hline 
CPU-core frequency & 2.60GHz  & c)\\ 
\hline 
Tick/tok & It's a Tick model (Ivy Bridge, shrink to 22nm) & d) \\ 
\hline 
Cycles/issue for floating point additions & 1 add/cycle & e) \\ 
\hline 
Cycles/issue for floating point multiplications & 1 mul/cycle & f) \\ 
\hline 
Max. theoretical float peak performance & 2 flops/cycle and 5.2 Gflop/s & g) \\ 
\hline 
\end{tabular} 
\\
Source of Information:
\begin{itemize}
  \item \url{http://www.intel.com/content/dam/doc/manual/64-ia-32-architectures-optimization-manual.pdf} Page 2-37
  \item \url{http://ark.intel.com/products/64891/Intel-Core-i7-3720QM-Processor-6M-Cache-up-to-3_60-GHz}
  \item ``sysctl -n machdep.cpu.brand\_string''
\end{itemize}
\end{document}